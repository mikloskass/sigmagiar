%Proposal Upload must be a .pdf file.
% On the upload form, for Title - enter a brief file name such as your last name + "Grant Proposal"
% Proposal upload file should include the following (in order)
% Project Title and Applicant's Name
% 500 Word Proposal Narrative with no more than 2 images embedded in the narrative
% Project Budget
% Literature Cited
\documentclass[11pt,a4paper,oneside]{article}
\usepackage[english,activeacute]{babel}
\usepackage[utf8]{inputenc}
\usepackage[scale=0.75]{geometry}
\usepackage{ragged2e}
\usepackage[biblabel,]{cite}

% \usepackage[numbers,sort]{natbib}
\usepackage{gensymb}
\usepackage{cmbright}
\usepackage{graphicx}
% \usepackage{sidecap}
\usepackage[hidelinks]{hyperref}
\renewcommand{\rmdefault}{phv}
\renewcommand{\sfdefault}{phv}
\SetSymbolFont{largesymbols}{normal}{OMX}{iwona}{m}{n}
\setcounter{secnumdepth}{0}
% \renewcommand{\thebibliographyname}{Literature cited}
\renewcommand\citeleft{(}
\renewcommand\citeright{)}
\renewcommand{\refname}{Literature Cited}
% \section*{\refname}
%------------------------------------

\title{Water Quality and Trophic Ecologic from Talampaya river (La Rioja, Argentina)}
\date{}
\author{Melina Celeste Crettaz Minaglia}

\begin{document}
% Project Title and Applicant's Name

% 500 Word Proposal Narrative with no more than 2 images embedded in the narrative PONER Dos IMAGENES
% Key to evaluation criteria.
% Big Picture: How well did the applicant demonstrate an understanding of the big picture/importance of the proposed investigation?
% Hypothesis: How well was the hypothesis or research question stated?
% Methodology: How well was the methodology described?
% Budget: How well were expenses, supplies, and equipment justified?
% Appropriateness: How well did the proposal fit within the scope of the GIAR program?  Does the study conform to ethical standards?
% Competitiveness: How competitive was this application in comparison to other proposals in this field?
\maketitle
\subsection*{Proposed Investigation}
The project aims to study the elements of ecosystems from permanent riverbed Talampaya river and to stablish the trophic relationships between its elements in order to increase the knowledge about FUNCIONAMIENTO of system and MEJORAR la conservacion del sitio. 

\bigskip
\begin{figure}[h] %this figure will be at the right
\centering
\includegraphics[height=5cm]{ranita.png}
\caption{The Laguna Raimunda Frog, \textit{Atelognathus reverberii}. Photograph by Nicolas Kass.}
\end{figure}
\bigksip

The Talampaya National Park...

Annual average rain is 150-170 mm but only in rainy season. This generates temporary watercourses and strong erosions in the soils. Despide scarse hidrical resources, no information is available about water quality nor asociated fauna in the National Park Talampaya. 

Talampaya river is a main water courses that it has performed the Cañon de Talampaya and Los Cajones. In this place, the watercourse is not permanent and it will find ponds and small threads of water. It is very singular duo to the presence of water inside of desertic environment (cita paper), therefore it can be found endemic species such as \textit{Liolaemus talampaya} (Personal observation). Restricted access allows the presence of emblematic species such as \textit{Puma concolor}, \textit{Vultur gryphus} and \textit{Lama guanicoe} (Personal communication).

Hipotesis: en el rio Talampaya se encuentran especies de anfibios y macroinvertebrados no documentados previamente y estas son fundamentales para los eslabones troficos superiores. 

In order to fulfill the objective, a sampling will be done in the permanent riverbed Talampaya river. The QGIS free Software will be used to characterize the basin morphometrically (Gaspari et al.,2013). The parameters axial length, perimeter, area and average width, Gravenius form and coeficient Gravenius compacity will be measured (Londoño Arango, 2001). In addition, relief and drainage parameters such as drainage density, drainage length, lenght of the main course, average pending and concentration time (Gaspari et al., 2013). Flow wiill be measured according to Sánchez et al., (2010) by float methods. In the lenght watercourse, parameters of water quality such as temperature, disolved oxygen, electric conductivity and total disolved solids will be measured in order to know the spatial gradient of these parameters. Moreover, water samples will be collected by determine nutrients of nitrogen and phosphorus and quimic demand of oxygen (APHA,xxxx) in order to estimate food available of microorganisms (heterotrophic and autotrophic). In addition, chlorophyll-a as (Marker et al., 1980) biomass estimator will be determine in order to estimate primary productors. Ionic composition will be measured in order to CLASIFICAR the water. 
Macroinvertebrates will be collected using the Surber net in the lenght of watercourse (Darrigran et al., 2007). Organisms will be identified until MENOR taxonomic level (Dominguez y Fernandez, 2009) and calculated richness and diversity parameters according to Pla (2006), functional food groups and ecosystemic atributtes (Cummins et al., 2005). Finally, amphibia will be collected with manual net exploring differents microhabitats in Talampaya rivers and estimated the richness.

%The budget covers the rent of a 4-WD vehicle, the supplies and the fuel costs required to get to the Plateau. This trip will not assess activities that need the handling of live animals nor sampling (chytrid fungus, stomach-flushing, etc.), but only the visual recognition and location recording. The priority of this study is to define where this species is distributed throughout the plateau by registering the presence of individuals found on the different lagoons of the Plateau and their surroundings and then model this distribution through MaxEnt \cite{phillips2006}. Besides, the collection of this important data will allow the development of different GIS-based studies, such as those which involve the climate-change influence on the species distribution prediction \cite{ihlow2016} and the definition of a conservation status of the species.

%Although this species is not categorized as endangered, the distribution range of the species is not accurately assessed and is significantly smaller than the currently proposed. The main reason for this is that these frogs are only found up to 90 feet far from the lagoons \cite{cei1969}. Moreover, the presence of frogs is not associated to all of the lagoons found in the area (\textit{pers. obs.}). Therefore this species is not present on all the lagoons of the plateau, and we want to define specifically in which ones in order to pursue the different studies involved in my Ph.D. project, to provide novel base knowledge about the species and to promote its habitat conservation.
\vfill
\begin{figure}[h] %this figure will be at the right
\centering
\includegraphics[height=10cm]{lugar.png}
\caption{At left, location of the plateau in Argentina. At right, dominant landscape and typical lagoon where the species is found. Photographs by Nicolas Kass and map from Google Inc.}
\end{figure}
\vfill


% \bibliographystyle{unsrt}
% \fontsize{9}{9}\selectfont
% \bibliography{bib}

%---------------

\newpage

\subsection*{Budget}

\underline{Project budget:} \$1000

\bigskip

\noindent \underline{Items requested from Sigma Xi:}

\begin{description} %completar
	\item[\$350] Fuel (XXXX miles @ \$0.23 per mile)
	\item[\$240] Water samples analysis: anoins, cations, dqo and nutrients
	\item[\$60] Supplies for 5 days
	\item[\$40] Camping for 5 days
	\item[\$15] Samples bottles
\end{description}


\newpage

\section*{Literature Cited}
\begingroup
\renewcommand{\section}[2]{}%
\begin{thebibliography}{6}

\bibitem{cei1969}
	José Miguel Alfredo María Cei.
	The patagonian telmatobiid fauna of the volcanic Somuncurá plateau of Argentina.
	\textit{Journal of Herpetology},
	pages 1–18,
	1969.

\bibitem{akmentis2015}
	Mauricio Sebastián Akmentins, Melina Alicia Velasco, Camila Alejandra Kass, and Federico Pablo Kacoliris. A new threat for the endangered frog \textit{Atelognathus reverberii}
(anura: Batrachylidae) in argentinean Patagonia.
	\textit{Phyllomedusa: Journal of Herpetology},
	14(1):63–66,
	2015.

\bibitem{arellano2015}
	Marı́a Luz Arellano, Mauricio Akmentis, Melina Alicia Velasco, Camila Kass, and Kacoliris Federico Pablo. First report of batrachochytrium dendrobatidis in \textit{Atelognathus reverberii}, a threatened species in Argentina.
	\textit{Herpetological Review},
	46:354 – 356,
	2015.

\bibitem{iucn2015}
	IUCN SSC Amphibian Specialist Group.
	2016.
	\textit{Atelognathus reverberii}.
	The IUCN Red List of Threatened Species 2016: e.T2297A85301943.
	\url{http://dx.doi.org/10.2305/IUCN.UK.2016-1.RLTS.T2297A85301943.en}.
	Downloaded on 29 September 2017.

\bibitem{phillips2006}
	Steven J Phillips, Robert P Anderson, and Robert E Schapire.
	Maximum entropy modeling of species geographic distributions.
	\texit{Ecological modelling},
	190(3): 231–259,
	2006.
\bibitem{ihlow2016}
	Flora Ihlow, Julien Courant, Jean Secondi, Anthony Herrel, Rui Rebelo, G John Measey,
Francesco Lillo, F André De Villiers, Solveig Vogt, Charlotte De Busschere, et al.
	Impacts of climate change on the global invasion potential of the african clawed frog \texit{Xenopus laevis}.
	\textit{PloS one},
	11(6):e0154869,
	2016.

\end{thebibliography}
\endgroup

\end{document}
